\documentclass[12pt, letterpaper]{article}
% Reduce the page margins to 3/4 of an inch
\usepackage[margin=0.75in]{geometry}

\usepackage{bookmark}

\usepackage{enumitem}

% References
\usepackage{hyperref}

% Import citations for use with BibTeX
\usepackage{cite}

\usepackage[toc,page]{appendix}
% Table formatting
\usepackage{tabularx}

% Paragraph-blocks, allowing multi-line cells in tables.
\usepackage{makecell}

\usepackage{float}

% Source code
\usepackage{listings}

% Allows float barriers
\usepackage{placeins}


\newcommand{\quickfigure}[3]{
  \FloatBarrier
  \begin{figure}[!h]
    \centering
    \includegraphics[width=#1\linewidth]{#2}
    \caption{#3}
  \end{figure}
  \FloatBarrier
}

% Restrict table of contents to parts through subsubsections
\setcounter{tocdepth}{3}


\usepackage{fancyhdr}
\pagestyle{fancy}
\fancyhf{}
\rhead{Page \thepage}
\lhead{Software Design Specification for Flight Plan Editor}

\makeatletter
\renewcommand{\maketitle}{\bgroup\setlength{\parindent}{0pt}
\thispagestyle{empty}
\null
  \begin{flushleft}
  \vspace{15mm}
  \vskip2mm
  \Huge{\textbf{\@title}}
  \vspace{8cm}

  \textbf{Group: CIOS Digital}
  \@author
  \end{flushleft}\egroup
}
\makeatother

\title{Software Design Specification\\for\\Flight Plan Editor}
\date{}
\author{
\\Cedrick Cooke\\
Ian Littke\\
Owen Roth-Lerner\\
Sander Scherman Garzon
}



% \title{Software Design Document \\
% Flight Planning
% }
% \author{Cedrick Cooke
%     \and Ian Littke
%     \and Owen Roth-Lerner
%     \and Sander Scherman Garzon
% }
% \date{Version 0.1 \\ 2017-04-13}

\begin{document}
\maketitle
\newpage
\section*{Revision History}
\begin{tabularx}{\textwidth}{|l|r|X|l|}
\hline
\textbf{Name} & \textbf{Date} & \textbf{Reason for Change} & \textbf{Version} \\ \hline
Various & 2017-04-13 & Initial document body and writing & Draft 1 \\ \hline
        &            &                                   &         \\ \hline
        &            &                                   &         \\ \hline
        &            &                                   &         \\ \hline
        &            &                                   &         \\ \hline
        &            &                                   &         \\ \hline
        &            &                                   &         \\ \hline
        &            &                                   &         \\ \hline
        &            &                                   &         \\ \hline
        &            &                                   &         \\ \hline
        &            &                                   &         \\ \hline
        &            &                                   &         \\ \hline
        &            &                                   &         \\ \hline
        &            &                                   &         \\ \hline
        &            &                                   &         \\ \hline
        &            &                                   &         \\ \hline
        &            &                                   &         \\ \hline
        &            &                                   &         \\ \hline
        &            &                                   &         \\ \hline
\end{tabularx}
\newpage
\tableofcontents

\newpage


\section{Introduction}
  \subsection{Purpose}
    This software design document describes the architecture and system design of the CIOS Digital Flight Planning Editor release 1.0.
    The intended audience for this software will be members of the Civilian Air Patrol nationwide who will design and load flight plans to be mounted onto a Garmin G1000 equipped aircraft.

\subsection{Scope}
  The Flight Planning Editor will allow flight planners in the Civilian Air Patrol (CAP) to plan their flights in an easy fashion and store them in a Secure Digital (SD) card. A detailed project description is available in the Flight Plan Editor Vision and Scope Document. The section in that document titled ”Scope of Initial and Subsequent Releases” lists the features that are scheduled for full or partial implementation in this release.

  \subsection{Intended Audience}

  \subsection{References}

  \subsection{Definitions, Acronyms, and Abbreviations}

\section{System Overview}
Give a general description of the functionality, context and design of your project. Provide any background information if necessary.
  \subsection{System context Diagram}

\section{Design Considerations}
  \subsection{Assumptions and Dependencies}
  \subsection{General Constraints}
  \subsection{Goals and Guidelines}
  \subsection{Development Methods}

\section{Architectural Strategies}
\section{System Architecture}
  \subsection{Architectural Design}
  \subsection{Decomposition Description}
  \subsection{Design Rationale}

\section{Detailed System Design}
  lots of stuff goes here....

\section{Appendex}
  \subsection{Glossary}
  \subsection{Data Dictionary}



\bibliography{citations}{}
\bibliographystyle{plain}
\end{document}
