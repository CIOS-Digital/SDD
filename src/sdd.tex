\documentclass[12pt, letterpaper]{article}
% Reduce the page margins to 3/4 of an inch
\usepackage[margin=0.75in]{geometry}

\usepackage{bookmark}

\usepackage{enumitem}

% References
\usepackage{hyperref}

% Import citations for use with BibTeX
\usepackage{cite}

% Table formatting
\usepackage{tabularx}

% Paragraph-blocks, allowing multi-line cells in tables.
\usepackage{makecell}

\usepackage{float}

% Source code
\usepackage{listings}


\newcommand{\quickfigure}[3]{
  \begin{figure}[!ht]
    \centering
    \includegraphics[width=#1\linewidth]{#2}
    \caption{#3}
  \end{figure}
}

% Restrict table of contents to parts through subsubsections
\setcounter{tocdepth}{3}


\title{Software Design Document \\
Flight Planning
}
\author{ Cedrick Cooke
    \and Ian Littke
    \and Owen Roth-Lerner
    \and Sander Scherman Garzon
}
\date{Version 0.1 \\ 2017-04-13}

\begin{document}
\maketitle

\tableofcontents

\section*{Summary of Changes}
\begin{tabularx}{\textwidth}{|l|r|X|l|}
\hline
Editor & Revision & Description & Date \\ \hline \hline
Various & 0.1 & Initial document body and writing & 2017-04-13 \\ \hline
\end{tabularx}

\section{Requirements}
\subsection{Background}
One of the missions which the Civil Air Patrol may have to complete in the event of a serious earthquake is to survey bridge infrastructure to verify its structural integrity.  Pilots are sent on missions to fly over and photograph bridges across the state of Washington.  Currently, CAP pilots are given notebooks containing flight plans with a series of geographic coordinates.  While flying, the pilots manually key the coordinates of their next location into the cockpit navigation system.  The Flight Plan Editor software will allow CAP to generate flight plan datafiles that are compatible with the navigation system so that pilots can select them from the cockpit navigation menu, thereby relieving them of the responsibility of manualy entering location coordinates.



\bibliography{citations}{}
\bibliographystyle{plain}
\end{document}
