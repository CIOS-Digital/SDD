\documentclass[12pt, letterpaper]{article}
\usepackage{graphicx}
\graphicspath{ {images/} }
% Reduce the page margins to 3/4 of an inch
\usepackage[margin=0.75in]{geometry}

\usepackage{bookmark}

\usepackage{enumitem}

% References
\usepackage{hyperref}

% Import citations for use with BibTeX
\usepackage{cite}

% Table formatting
\usepackage{tabularx}

% Paragraph-blocks, allowing multi-line cells in tables.
\usepackage{makecell}

\usepackage{float}

% Source code
\usepackage{listings}


\newcommand{\quickfigure}[3]{
  \begin{figure}[!ht]
    \centering
    \includegraphics[width=#1\linewidth]{#2}
    \caption{#3}
  \end{figure}
}

% Restrict table of contents to parts through subsubsections
\setcounter{tocdepth}{3}


\usepackage{fancyhdr}
\pagestyle{fancy}
\fancyhf{}
\rhead{Page \thepage}
\lhead{Software Design Specification for Flight Plan Editor}
\newcolumntype{R}{>{\raggedleft\arraybackslash}X}%

\makeatletter
\renewcommand{\maketitle}{\bgroup\setlength{\parindent}{0pt}
\thispagestyle{empty}
\null
  \begin{flushleft}
  \vspace{15mm}
  \vskip2mm
  \Huge{\textbf{\@title}}
  \vspace{8cm}

CSCI 492: Spring 2017\\
  \textbf{Group: CIOS Digital}\\

  \@author
  \end{flushleft}\egroup
}
\makeatother

\title{Software Design Specification\\for\\Flight Plan Editor}
\date{}
\author{
Cedrick Cooke\\
Ian Littke\\
Owen Roth-Lerner\\
Sander Scherman Garzon
}



% \title{Software Design Document \\
% Flight Planning
% }
% \author{Cedrick Cooke
%     \and Ian Littke
%     \and Owen Roth-Lerner
%     \and Sander Scherman Garzon
% }
% \date{Version 0.1 \\ 2017-04-13}

\begin{document}
\maketitle

\newpage
\section*{Revision History}
\begin{tabularx}{\textwidth}{|l|r|X|l|}
\hline
\textbf{Name} & \textbf{Date} & \textbf{Reason for Change} & \textbf{Version} \\ \hline
Various & 2017-04-13 & Initial document body and writing & Draft 1 \\ \hline
        &            &                                   &         \\ \hline
        &            &                                   &         \\ \hline
        &            &                                   &         \\ \hline
        &            &                                   &         \\ \hline
        &            &                                   &         \\ \hline
        &            &                                   &         \\ \hline
        &            &                                   &         \\ \hline
        &            &                                   &         \\ \hline
        &            &                                   &         \\ \hline
        &            &                                   &         \\ \hline
        &            &                                   &         \\ \hline
        &            &                                   &         \\ \hline
        &            &                                   &         \\ \hline
        &            &                                   &         \\ \hline
        &            &                                   &         \\ \hline
        &            &                                   &         \\ \hline
        &            &                                   &         \\ \hline
        &            &                                   &         \\ \hline
\end{tabularx}
\newpage
\tableofcontents

\newpage


\section{Introduction}
  \subsection{Purpose}
    This software design document describes the architecture and system design of the CIOS Digital Flight Planning Editor release 1.0.
    The intended audience for this document will be primarily the development team at CIOS Digital working on the Flight Planning Editor and secondary any members of the Civilian Air Patrol nationwide who will design and load flight plans to be mounted onto a Garmin G1000 equipped aircraft.

  \subsection{Scope}
    The Flight Planning Editor will allow flight planners in the Civilian Air Patrol (CAP) to plan their flights in an easy fashion and store them in a Secure Digital (SD) card. A detailed project description is available in the Flight Plan Editor Vision and Scope Document. The section in that document titled ”Scope of Initial and Subsequent Releases” lists the features that are scheduled for full or partial implementation in this release.
    This document provides the artchitecture and design of Release 1.0 of the software.

  % \subsection{Overview}
  %   Section \ref{system} of this document details the system overview and gives a description of the functionality. \\
  %   Section \ref{design} covers the system architecture including the architectural design, decomposition description, and design rationale. \\
  %   Section 4 details the data design giving a description and data dictionary. \\
  %   Section 5 is the component design, offering a comprehensive look at each component and how they link with one another. \\
  %   Section \ref{ui} includes the human interface design. \\
  %   Section 7 lists the Appendicies. \\
  \subsection{Reference Material}
    \begin{itemize}
      \setlength{\itemsep}{1pt}
      \setlength{\parskip}{0pt}
      \setlength{\parsep}{0pt}
      \item CIOS Digital Vision and Scope Document
      \item CIOS Digital Software Requirements Specification (SRS)
    \end{itemize}

  \subsection{Definitions, Acronyms, and Abbreviations}
  	\begin{tabularx}{\textwidth}{|l|R|} \hline
    	Acronym & Definition \\ \hline
    	CAP & Civilian Air Patrol  \\ \hline
    	SD & Secure Digital (Memory card) \\ \hline
  	\end{tabularx}

  \section{System Overview}\label{system}
    The CAP will need to create flight plans based on existing structures so the application will focus around a map.
    The user will then be able to either click on map to insert a new waypoint in the flight plan or manually input in a latitude/longitude coordinate system.
    Both are included to allow precise input of key waypoints (such as an overpass) or coarse input for secondary waypoints (such as an entry/depature course).
    The User Interface is furter explained in Section \ref{sec:ui}.

    \begin{enumerate}
      \item UC:Show Help
	\begin{figure}[!ht]
		\caption{Show Help Use Case}
    		\centering
    		\includegraphics{showhelp}
	\end{figure}
	If a user is experiencing any difficulty with regards to using the application, there will be an included help page to assist with basic knowledge and instructions on how to use the program. The 		user must click the 'help' tab located at the top of the program which will open a new window offering descriptions and a quick tutorial on how to use the application.

      \item UC:Create New Plan
	\begin{figure}[!ht]
		\caption{Create New Plan Use Case}
    		\centering
    		\includegraphics{newplan}
	\end{figure}
	In order to create a new flight plan, the user must click on 'New Flight Plan.' Doing so will first initiate a check whether or not a current plan is in use, and if so, check to see whether or not 		this current plan is saved before moving on. Following this initial step, the program will proceed by clearing all waypoints currently drawn on the map. Next, the program will prompt the user to 		enter a new name for the flight plan, so long as this name is valid, the new flight plan will have been created with that given name and is now ready for new waypoints to be added. If the name is 		invalid, either wait for the user to enter in a valid name, or allow the user to cancel the operation.

      \item UC:Save Plan
	\begin{figure}[!ht]
		\caption{Save Plan Use Case}
    		\centering
    		\includegraphics{saveplan}
	\end{figure}
	When a user wants to save the current flight plan, they will select the 'save' button and be given the option to save under a new name, or continuing with the existing save name should one exist. 		Once a valid name is entered, the program will generate the current flight plan and format it to the same XML standard which will be saved onto the SD card, except at a given target on the 		computer desktop. 

      \item UC:Edit Plan
	\begin{figure}[!ht]
		\caption{Edit Plan Use Case}
    		\centering
    		\includegraphics{editplan}
	\end{figure}
	In order to edit a previously saved flight plan, the program must first make sure that the user does not currently have a flight plan open that has not been saved. If there is a plan that needs to be saved, the program saves the program to an XML document and then clears the map API and list interface, and loads the other plan’s waypoints from the XML to the list and updates the map.

      \item UC:Add Waypoint
	\begin{figure}[!ht]
		\caption{Add Waypoint Use Case}
    		\centering
    		\includegraphics{add}
	\end{figure}
	To add a waypoint, the user must first enter both a longitude and lattitude coordinate, once completed, click the 'add waypoint' button located on the upper left portion of the application window. 		In doing so, the application will first validate whether or not the input is valid, and if so, the application will call upon it's add waypoint function. While doing this, the application will 		also draw a cooresponding point on the visual map, as well as a arrow connecting the newly added waypoint to the last waypoint currently drawn on the map, this step is ignored if it is the first 		waypoint entered. Should the input be invalid, it would alert the user that an incorrect coordinate was entered and the program does not enter the invalid data into the flight editor.

      \item UC:Remove Waypoint
	\begin{figure}[!ht]
		\caption{Remove Use Case}
    		\centering
    		\includegraphics{remove}
	\end{figure}
	If a user wants to remove a waypoint from the flight plan, they must first click on the 'x' to delete the waypoint. Doing so initiates a series of checks which the program computes automatically, such as seeing whether or not the waypoint is between other waypoints. Depending on the placement of this waypoint in question and if it is preceded or followed by another waypoint, the program will accordingly remove and redraw the connecting arrows connecting the preceding and following waypoints. Once complete, the program will remove the waypoint.

      \item UC:Rearrange Waypoint
	\begin{figure}[!ht]
    		\centering
		\caption{Rearrange Waypoint Use Case}
    		\includegraphics{rearrange}
	\end{figure}
	The user must be able to not only create new waypoints and delete them from a list, but be able to shift existing waypoints around. In order to re-arrange the order of flight waypoints in the flight plan list, the user must select a specific waypoint and then decide whether they want to move it up in order or down in order. Depending on which way the user decides to shift the waypoint, the list will update and display the new order. Afterwhich the program will shift the arrows between waypoints over the google maps API. 

      \item UC:Map Toggle
	\begin{figure}[!ht]
		\caption{Map Toggle Use Case}
    		\centering
    		\includegraphics{maptoggle}
	\end{figure}
	The user may wish to shift which kind of map the flight plan display is using. If the user wishes to changed from the default road map background to a terrain, satellite, or hybrid map, the user will simply select one of the four map “buttons” at which point the google map API will simply change the background that the points are overlaid on. The only checks at this point is the redundancy check of making sure that the user is not selecting the already-selected map. 

    \end{enumerate}
    %  \subsection{System context Diagram}

\section{Design Considerations} \label{dsign}
  \subsection{Assumptions and Dependencies}
    See CIOS Digital Software Requirements Specification Section 2.7 for details.
  \subsection{General Constraints}
    See CIOS Digital Software Requirements Specification Section 4 for details.
  \subsection{Goals and Guidelines}
    \begin{itemize}
      \setlength{\itemsep}{1pt}
      \setlength{\parskip}{0pt}
      \setlength{\parsep}{0pt}
      \item Emphasis shall be placed on usability to make flight plans easily.
      \item The tiled map must have the proper zoom level as well as low-latency.
      \item The user must be able to edit as well as select pre-saved flight plans to an SD-card.
      \item The user must be able to switch different map overlays.
      \item The saved flight plans must be able to be loaded by a Garmin G1000 integrated flight instrument system.
    \end{itemize}
  \subsection{Development Methods}
    This project is being conducted using a loose agile paradigm with biweekly status meetings.
    One of these meetings coincides with meeting with our faculty advisor.
    Project code will be stored in a git repository with code reviews before integration into the master branch.

% \section{Architectural Strategies}
% \section{System Architecture}
%   \subsection{Architectural Design}
%   \subsection{Decomposition Description}
%   \subsection{Design Rationale}
\newpage

\section{Human Interface Design} \label{sec:ui}

\subsection{Overview of User Interface}
The interface for the Flight Planning Editor will be a Windows desktop application with a design concept shown in the illustration below.
	\subsection{Screen Image}
    \begin{figure}[h!]
      \includegraphics[width=0.95\linewidth]{figs/FlightPlanning_Interface}\caption{Mockup of User Interface}
    \end{figure}
	\subsection{Screen Objects and Actions}
In the above window there are effectively four subsections to the interface. In the upper left corner, where it reads 'Lat', this portion of the interface is where the user can input both the lattitude and longitude coordinates as input into the flight editor. To add these coordinates to the current flight plan, simply click on the 'Add Waypoint' box and the program will create a new point on the map as well as create an addition into the Order of Points Visited section detailed below. \\
Directly below, there will be another section which reads the various flight plans that are currently loaded and their respective names (i.e. Flight Plan A). To distinguish which flight plan the user wants to current edit, that flight plan will have a box which shows an 'X' or is checked to denote that it is currently selected. \\
Underneath the above section, in the lower left corner of the application, there is a section for the order of points visited. In this section there will be a list of all the currently added waypoints named 'W1', 'W2'... Each of these waypoints are listed in order from top to the bottom. To the right of these names there are arrows pointing upwards and downwards, and by clicking on this icon up or down, the user will be able to re-arrange the order in which each point is visited and the application will swap the waypoint boxes in this section, as well as redraw the arrow on the map to reflect the changes made. In the upper left portion of each boxes for the waypoints will be a small box with an 'X' when clicked will delete the waypoint and remove it from the map.\\
The largest portion of the interface will display the map. This section will display the current area that the user has selected by panning and zooming in or out of the map. Inside there will be each of the waypoints listed and arrows denoting the direction of the flight plan. As the user enters a new lattitude and longitude, these waypoints will be refreshed and shown ontop of the map in their respective locations. \\
Also included, however not displayed in the diagram are the file and help pull down menus. Within the file menu, there will be the option to start a new or open an existing flight plan. The ability to save the flight plan, or to export onto the SD card. As well as being able to exit the application. Under the help menu, there will be a tutorial to showcase the basic functionality and an included about page with the company information and the version number of the application.

\appendix
\section{Remove Me!}
\begin{figure}[!ht]
    \centering
    \includegraphics{example-diagram}
\end{figure}
This is an example of adding a diagram with plantuml.
Make sure you have an appropriately named UML file in the plantuml folder,
and just use includegraphics.


% \bibliography{citations}{}
% \bibliographystyle{plain}
\end{document}
