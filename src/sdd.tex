\documentclass[12pt, letterpaper]{article}
\usepackage{graphicx}
\graphicspath{ {images/} }
% Reduce the page margins to 3/4 of an inch
\usepackage[margin=0.75in]{geometry}

\usepackage{bookmark}

\usepackage{enumitem}

% References
\usepackage{hyperref}

% Import citations for use with BibTeX
\usepackage{cite}

% Table formatting
\usepackage{tabularx}

% Paragraph-blocks, allowing multi-line cells in tables.
\usepackage{makecell}

\usepackage{float}

% Source code
\usepackage{listings}


\newcommand{\quickfigure}[3]{
  \begin{figure}[!ht]
    \centering
    \includegraphics[width=#1\linewidth]{#2}
    \caption{#3}
  \end{figure}
}

% Restrict table of contents to parts through subsubsections
\setcounter{tocdepth}{3}


\usepackage{fancyhdr}
\pagestyle{fancy}
\fancyhf{}
\rhead{Page \thepage}
\lhead{Software Design Specification for Flight Plan Editor}

\makeatletter
\renewcommand{\maketitle}{\bgroup\setlength{\parindent}{0pt}
\thispagestyle{empty}
\null
  \begin{flushleft}
  \vspace{15mm}
  \vskip2mm
  \Huge{\textbf{\@title}}
  \vspace{8cm}

  \textbf{Group: CIOS Digital}
  \@author
  \end{flushleft}\egroup
}
\makeatother

\title{Software Design Specification\\for\\Flight Plan Editor}
\date{}
\author{
\\Cedrick Cooke\\
Ian Littke\\
Owen Roth-Lerner\\
Sander Scherman Garzon
}



% \title{Software Design Document \\
% Flight Planning
% }
% \author{Cedrick Cooke
%     \and Ian Littke
%     \and Owen Roth-Lerner
%     \and Sander Scherman Garzon
% }
% \date{Version 0.1 \\ 2017-04-13}

\begin{document}
\maketitle
\newpage
\section*{Revision History}
\begin{tabularx}{\textwidth}{|l|r|X|l|}
\hline
\textbf{Name} & \textbf{Date} & \textbf{Reason for Change} & \textbf{Version} \\ \hline
Various & 2017-04-13 & Initial document body and writing & Draft 1 \\ \hline
        &            &                                   &         \\ \hline
        &            &                                   &         \\ \hline
        &            &                                   &         \\ \hline
        &            &                                   &         \\ \hline
        &            &                                   &         \\ \hline
        &            &                                   &         \\ \hline
        &            &                                   &         \\ \hline
        &            &                                   &         \\ \hline
        &            &                                   &         \\ \hline
        &            &                                   &         \\ \hline
        &            &                                   &         \\ \hline
        &            &                                   &         \\ \hline
        &            &                                   &         \\ \hline
        &            &                                   &         \\ \hline
        &            &                                   &         \\ \hline
        &            &                                   &         \\ \hline
        &            &                                   &         \\ \hline
        &            &                                   &         \\ \hline
\end{tabularx}
\newpage
\tableofcontents

\newpage


\section{Introduction}
  \subsection{Purpose}
    This software design document describes the architecture and system design of the CIOS Digital Flight Planning Editor release 1.0.
    The intended audience for this software will be members of the Civilian Air Patrol nationwide who will design and load flight plans to be mounted onto a Garmin G1000 equipped aircraft.

\subsection{Scope}
  The Flight Planning Editor will allow flight planners in the Civilian Air Patrol (CAP) to plan their flights in an easy fashion and store them in a Secure Digital (SD) card. A detailed project description is available in the Flight Plan Editor Vision and Scope Document. The section in that document titled ”Scope of Initial and Subsequent Releases” lists the features that are scheduled for full or partial implementation in this release.

  \subsection{Intended Audience}
This documented is to be used primarily by the developement team at CIOS Digital working on the Flight Planning Editor.

  \subsection{Overview}
Section 2 of this document details the system overview and gives a description of the functionality. \\
Section 3 covers the system architecture including the architectural design, decomposition description, and design rationale. \\
Section 4 details the data design giving a description and data dictionary. \\
Section 5 is the component design, offering a comprehensive look at each component and how they link with one another. \\
Section 6 includes the human interface design. \\
Section 7 Appendicies \\

  \subsection{Definitions, Acronyms, and Abbreviations}
	\begin{tabularx}{\textwidth}{|l|l} \hline
	Acronym & Definition \\ \hline \hline
	CAP & Civilian Air Patrol  \\ \hline
	SD & Secure Digital (Memory card) \\ \hline
	\end{tabularx}

\section{System Overview}
Give a general description of the functionality, context and design of your project. Provide any background information if necessary.
  \subsection{System context Diagram}

\section{Design Considerations}
  \subsection{Assumptions and Dependencies}
  \subsection{General Constraints}
  \subsection{Goals and Guidelines}
  \subsection{Development Methods}

\section{Architectural Strategies}
\section{System Architecture}
  \subsection{Architectural Design}
  \subsection{Decomposition Description}
  \subsection{Design Rationale}

\section{Human Interface Design}

\subsection{Overview of User Interface}
The interface for the Flight Planning Editor will be a windows desktop application with a design concept shown in the illustration below. 
	\subsection{Screen Image}
\includegraphics{FlightPlanning_Interface}
	\subsection{Screen Objects and Actions}
In the above window there are effectively four subsections to the interface. In the upper left corner, where it reads 'Lat', this portion of the interface is where the user can input both the lattitude and longitude coordinates as input into the flight editor. To add these coordinates to the current flight plan, simply click on the 'Add Waypoint' box and the program will create a new point on the map as well as create an addition into the Order of Points Visited section detailed below. \\
Directly below, there will be another section which reads the various flight plans that are currently loaded and their respective names (i.e. Flight Plan A). To distinguish which flight plan the user wants to current edit, that flight plan will have a box which shows an 'X' or is checked to denote that it is currently selected. \\
Underneath the above section, in the lower left corner of the application, there is a section for the order of points visited. In this section there will be a list of all the currently added waypoints named 'W1', 'W2'... Each of these waypoints are listed in order from top to the bottom. To the right of these names there are arrows pointing upwards and downwards, and by clicking on this icon up or down, the user will be able to re-arrange the order in which each point is visited and the application will swap the waypoint boxes in this section, as well as redraw the arrow on the map to reflect the changes made. In the upper left portion of each boxes for the waypoints will be a small box with an 'X' when clicked will delete the waypoint and remove it from the map.\\
The largest portion of the interface will display the map. This section will display the current area that the user has selected by panning and zooming in or out of the map. Inside there will be each of the waypoints listed and arrows denoting the direction of the flight plan. As the user enters a new lattitude and longitude, these waypoints will be refreshed and shown ontop of the map in their respective locations. \\
Also included, however not displayed in the diagram are the file and help pull down menus. Within the file menu, there will be the option to start a new or open an existing flight plan. The ability to save the flight plan, or to export onto the SD card. As well as being able to exit the application. Under the help menu, there will be a tutorial to showcase the basic functionality and an included about page with the company information and the version number of the application. 

\section{Appendex}
  \subsection{Glossary}
  \subsection{Data Dictionary}

\bibliography{citations}{}
\bibliographystyle{plain}
\end{document}
