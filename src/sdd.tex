\documentclass[12pt, letterpaper]{article}
\usepackage{graphicx}
\graphicspath{ {images/} }
% Reduce the page margins to 3/4 of an inch
\usepackage[margin=0.75in]{geometry}

\usepackage{bookmark}

\usepackage{enumitem}

% References
\usepackage{hyperref}

% Import citations for use with BibTeX
\usepackage{cite}

\usepackage[toc,page]{appendix}
% Table formatting
\usepackage{tabularx}

% Paragraph-blocks, allowing multi-line cells in tables.
\usepackage{makecell}

\usepackage{float}

% Source code
\usepackage{listings}

% Allows float barriers
\usepackage{placeins}


\newcommand{\quickfigure}[3]{
  \FloatBarrier
  \begin{figure}[!h]
    \centering
    \includegraphics[width=#1\linewidth]{#2}
    \caption{#3}
  \end{figure}
  \FloatBarrier
}

% Restrict table of contents to parts through subsubsections
\setcounter{tocdepth}{3}


\usepackage{fancyhdr}
\pagestyle{fancy}
\fancyhf{}
\rhead{Page \thepage}
\lhead{Software Design Specification for Flight Plan Editor}
\newcolumntype{R}{>{\raggedleft\arraybackslash}X}%

\makeatletter
\renewcommand{\maketitle}{\bgroup\setlength{\parindent}{0pt}
\thispagestyle{empty}
\null
  \begin{flushleft}
  \vspace{15mm}
  \vskip2mm
  \Huge{\textbf{\@title}}
  \vspace{8cm}

CSCI 492: Spring 2017\\
  \textbf{Group: CIOS Digital}\\

  \@author
  \end{flushleft}\egroup
}
\makeatother

\title{Software Design Specification\\for\\Flight Plan Editor}
\date{}
\author{
Cedrick Cooke\\
Ian Littke\\
Owen Roth-Lerner\\
Sander Scherman Garzon
}



% \title{Software Design Document \\
% Flight Planning
% }
% \author{Cedrick Cooke
%     \and Ian Littke
%     \and Owen Roth-Lerner
%     \and Sander Scherman Garzon
% }
% \date{Version 0.1 \\ 2017-04-13}

\begin{document}
\maketitle

\newpage
\section*{Revision History}
\begin{tabularx}{\textwidth}{|l|r|X|l|}
\hline
\textbf{Name} & \textbf{Date} & \textbf{Reason for Change} & \textbf{Version} \\ \hline
Various & 2017-04-13 & Initial document body and writing & Draft 1 \\ \hline
        &            &                                   &         \\ \hline
        &            &                                   &         \\ \hline
        &            &                                   &         \\ \hline
        &            &                                   &         \\ \hline
        &            &                                   &         \\ \hline
        &            &                                   &         \\ \hline
        &            &                                   &         \\ \hline
        &            &                                   &         \\ \hline
        &            &                                   &         \\ \hline
        &            &                                   &         \\ \hline
        &            &                                   &         \\ \hline
        &            &                                   &         \\ \hline
        &            &                                   &         \\ \hline
        &            &                                   &         \\ \hline
        &            &                                   &         \\ \hline
        &            &                                   &         \\ \hline
        &            &                                   &         \\ \hline
        &            &                                   &         \\ \hline
\end{tabularx}
\newpage
\tableofcontents

\newpage
\section{Introduction}
  \subsection{Purpose}
    This software design document describes the architecture and system design of the CIOS Digital Flight Planning Editor release 1.0.
    The intended audience for this document will be primarily the development team at CIOS Digital working on the Flight Planning Editor and secondary any members of the Civilian Air Patrol nationwide who will design and load flight plans to be mounted onto a Garmin G1000 equipped aircraft.

  \subsection{Scope}
    The Flight Planning Editor will allow flight planners in the Civilian Air Patrol (CAP) to plan their flights in an easy fashion and store them in a Secure Digital (SD) card. A detailed project description is available in the Flight Plan Editor Vision and Scope Document. The section in that document titled ”Scope of Initial and Subsequent Releases” lists the features that are scheduled for full or partial implementation in this release.
    This document provides the artchitecture and design of Release 1.0 of the software.

  % \subsection{Overview}
  %   Section \ref{system} of this document details the system overview and gives a description of the functionality. \\
  %   Section \ref{design} covers the system architecture including the architectural design, decomposition description, and design rationale. \\
  %   Section 4 details the data design giving a description and data dictionary. \\
  %   Section 5 is the component design, offering a comprehensive look at each component and how they link with one another. \\
  %   Section \ref{ui} includes the human interface design. \\
  %   Section 7 lists the Appendicies. \\
  \subsection{Reference Material}
    \begin{itemize}
      \setlength{\itemsep}{1pt}
      \setlength{\parskip}{0pt}
      \setlength{\parsep}{0pt}
      \item CIOS Digital Vision and Scope Document
      \item CIOS Digital Software Requirements Specification (SRS)
    \end{itemize}

  \subsection{Definitions, Acronyms, and Abbreviations}
  	\begin{tabularx}{\textwidth}{|l|R|} \hline
    	Acronym & Definition \\ \hline
    	CAP & Civilian Air Patrol  \\ \hline
    	SD & Secure Digital (Memory card) \\ \hline
  	\end{tabularx}

  \section{System Overview}\label{system}
    The CAP will need to create flight plans based on existing structures so the application will focus around a map.
    The user will then be able to either click on map to insert a new waypoint in the flight plan or manually input in a latitude/longitude coordinate system.
    Both are included to allow precise input of key waypoints (such as an overpass) or coarse input for secondary waypoints (such as an entry/depature course).
    The User Interface is furter explained in Section \ref{sec:ui}.

    \begin{enumerate}
      \item UC:Show Help
	\begin{figure}[!ht]
		\caption{Show Help Use Case}
    		\centering
    		\includegraphics{showhelp}
	\end{figure}
	If a user is experiencing any difficulty with regards to using the application, there will be an included help page to assist with basic knowledge and instructions on how to use the program. The 		user must click the 'help' tab located at the top of the program which will open a new window offering descriptions and a quick tutorial on how to use the application.

      \item UC:Create New Plan
	\begin{figure}[!ht]
		\caption{Create New Plan Use Case}
    		\centering
    		\includegraphics{newplan}
	\end{figure}
	In order to create a new flight plan, the user must click on 'New Flight Plan.' Doing so will first initiate a check whether or not a current plan is in use, and if so, check to see whether or not 		this current plan is saved before moving on. Following this initial step, the program will proceed by clearing all waypoints currently drawn on the map. Next, the program will prompt the user to 		enter a new name for the flight plan, so long as this name is valid, the new flight plan will have been created with that given name and is now ready for new waypoints to be added. If the name is 		invalid, either wait for the user to enter in a valid name, or allow the user to cancel the operation.

      \item UC:Save Plan
	\begin{figure}[!ht]
		\caption{Save Plan Use Case}
    		\centering
    		\includegraphics{saveplan}
	\end{figure}
	When a user wants to save the current flight plan, they will select the 'save' button and be given the option to save under a new name, or continuing with the existing save name should one exist. 		Once a valid name is entered, the program will generate the current flight plan and format it to the same XML standard which will be saved onto the SD card, except at a given target on the 		computer desktop. 

      \item UC:Edit Plan
	\begin{figure}[!ht]
		\caption{Edit Plan Use Case}
    		\centering
    		\includegraphics{editplan}
	\end{figure}
	In order to edit a previously saved flight plan, the program must first make sure that the user does not currently have a flight plan open that has not been saved. If there is a plan that needs to be saved, the program saves the program to an XML document and then clears the map API and list interface, and loads the other plan’s waypoints from the XML to the list and updates the map.

      \item UC:Add Waypoint
	\begin{figure}[!ht]
		\caption{Add Waypoint Use Case}
    		\centering
    		\includegraphics{add}
	\end{figure}
	To add a waypoint, the user must first enter both a longitude and lattitude coordinate, once completed, click the 'add waypoint' button located on the upper left portion of the application window. 		In doing so, the application will first validate whether or not the input is valid, and if so, the application will call upon it's add waypoint function. While doing this, the application will 		also draw a cooresponding point on the visual map, as well as a arrow connecting the newly added waypoint to the last waypoint currently drawn on the map, this step is ignored if it is the first 		waypoint entered. Should the input be invalid, it would alert the user that an incorrect coordinate was entered and the program does not enter the invalid data into the flight editor.

      \item UC:Remove Waypoint
	\begin{figure}[!ht]
		\caption{Remove Use Case}
    		\centering
    		\includegraphics{remove}
	\end{figure}
	If a user wants to remove a waypoint from the flight plan, they must first click on the 'x' to delete the waypoint. Doing so initiates a series of checks which the program computes automatically, such as seeing whether or not the waypoint is between other waypoints. Depending on the placement of this waypoint in question and if it is preceded or followed by another waypoint, the program will accordingly remove and redraw the connecting arrows connecting the preceding and following waypoints. Once complete, the program will remove the waypoint.

      \item UC:Rearrange Waypoint
	\begin{figure}[!ht]
    		\centering
		\caption{Rearrange Waypoint Use Case}
    		\includegraphics{rearrange}
	\end{figure}
	The user must be able to not only create new waypoints and delete them from a list, but be able to shift existing waypoints around. In order to re-arrange the order of flight waypoints in the flight plan list, the user must select a specific waypoint and then decide whether they want to move it up in order or down in order. Depending on which way the user decides to shift the waypoint, the list will update and display the new order. Afterwhich the program will shift the arrows between waypoints over the google maps API. 

      \item UC:Map Toggle
	\begin{figure}[!ht]
		\caption{Map Toggle Use Case}
    		\centering
    		\includegraphics{maptoggle}
	\end{figure}
	The user may wish to shift which kind of map the flight plan display is using. If the user wishes to changed from the default road map background to a terrain, satellite, or hybrid map, the user will simply select one of the four map “buttons” at which point the google map API will simply change the background that the points are overlaid on. The only checks at this point is the redundancy check of making sure that the user is not selecting the already-selected map. 

    \end{enumerate}
    %  \subsection{System context Diagram}

\newpage
\section{Design Considerations} \label{dsign}
  \subsection{Assumptions and Dependencies}
    See CIOS Digital Software Requirements Specification Section 2.7 for details.
  \subsection{General Constraints}
    See CIOS Digital Software Requirements Specification Section 4 for details.
  \subsection{Goals and Guidelines}
    \begin{itemize}
      \setlength{\itemsep}{1pt}
      \setlength{\parskip}{0pt}
      \setlength{\parsep}{0pt}
      \item Emphasis shall be placed on usability to make flight plans easily.
      \item The tiled map must have the proper zoom level as well as low-latency.
      \item The user must be able to edit as well as select pre-saved flight plans to an SD-card.
      \item The user must be able to switch different map overlays.
      \item The saved flight plans must be able to be loaded by a Garmin G1000 integrated flight instrument system.
    \end{itemize}
  \subsection{Development Methods}
    This project is being conducted using a loose agile paradigm with biweekly status meetings.
    One of these meetings coincides with meeting with our faculty advisor.
    Project code will be stored in a git repository with code reviews before integration into the master branch.

\newpage
\section{System Architecture}
\subsection{Architectural Strategies}
The architecture of the application follows general best-practices for an object-oriented Windows application,
  using the \emph{W}indows \emph{P}resentation \emph{F}oundation (WPF) framework from C\#.
\begin{itemize}
  \item The application splits source code into a series of projects within a Visual Studio solution.
    These projects are organized into an acyclic graph of dependencies to prevent bidirectional coupling wherever possible.
  \item The design follows the \emph{m}odel \emph{v}iew \emph{v}iew-\emph{m}odel (MVVM) pattern when applicable.
    This pattern encourages a strong separation between data, contained in the model, and the user inteface--or view.
    The view-model is often called the binding, and handles two-way communication between views and models;
    this is where most of the application's logic is handled.
    The MVVM pattern is considered the default for native Windows application using the e\emph{x}tensible \emph{a}pplication \emph{m}arkup \emph{l}anguage (XAML)\cite{msdn_mvvm}.
  \item Types should be as small as possible, and prefer composition to inheritence.
  \item When a type is representable with a small, finite number of distinct values, an enum shall be used instead of ambiguous types such as strings.
  \item Data should be immutable whenever possible.
  \item Map data will be provided through the Google Maps API.
    Google Maps was chosen as opposed to OpenStreetMap due to providing official support for downloading map tiles\cite{gmaps_tiles};
    OpenStreetMap tiles are provided by unofficial applications instead, and support varies between multiple options\cite{osm_tiles}.
  \item An internet connection shall be required during use of the application in order to stream map tile images as they become necessary.
    An alternative that was considered involved shipping an offline copy of the map with the application installer,
    but was decided against primarily because of large install sizes.
    Images will be cached locally to allow for reuse.
  \item A SQLite database will provide caching for settings and map images.
    SQLite is a natural fit as each application installation uses its own copy of an offline database, as opposed to a central database accessed online.
\end{itemize}
\subsection{Architectural Design}
The program is split into a series of five projects:
\begin{description}
  \item[Application] The user-facing application.
    This assembles the individual components and organizes much of the business logic.
  \item[Coordinate] Contains only a structure used to pair latitudes and longitudes.
  \item[FlightPlan] A library containing the flight planning logic.
    Primarily this serves as a wrapper around a collection of coordinates, providing a more strict set of invarients.
  \item[MapControl] A UI component library, keeping custom controls separate from the main application.
    Dependencies that are only used for implementation of the controls, but which do not effect their interface, are thus kept invisible to the final application.
  \item[MapDB] A library for interfacing with the Google Static Maps API.
    This library also provides SQLite caching for the requests.
\end{description}
The project dependencies can be seen below.
Note that they form a directed acyclic graph with a single common descendent: the application.
\quickfigure{0.5}{application-packages}{Dependency Graph of Projects within Solution}

\subsection{Decomposition Description}
\subsubsection{Coordinate}
\quickfigure{0.65}{coordinate}{UML Diagram of Coordinate Project}
Because flight planning software fundamentally marshals coordinates,
a base project containing the data structure is a common dependency throughout all other projects.
A coordinate is represented as a simple pair of latitude and longitude numbers.
The Coordinate type is an immutable structure containing little functional logic.
It implements equality, but is not ordered.

\subsubsection{Flight Plan}
\quickfigure{0.65}{flightplan}{UML Diagram of Flight Plan Project and Dependencies}
The flight plan project contains a single class, FlightPlan, responsible for providing an opaque wrapper around a collection of coordinates.
Functionality of the flight plan should allow editing of the route, as well as providing bindings necessary for XAML interoperation.
Logic for loading and saving the XML .fpl files is also contained in this class, using .NET's standard XML library.

\subsubsection{Mab Database}
\quickfigure{0.85}{mapdb}{UML Diagram of Map Database Project and Dependencies}
The map database provides a series of types externally visible only to the map controls.
The primary interface of this project is the MapProvider, an interface with a single method which accepts a MapImageSpec and returns a bitmap image.
Internally there is only one implementation of a MapProvider: the SQLiteDBMap.

A MapImageSpec is an immutable piece of data which correlates to the necessary arguments in a Google Static Maps query.
These elements are the map type, represented by a component MapType enumeration, a coordinate center, a decimal zoom level, and a width and height measured in pixels.
Although the width and height of the images are constant in the application's use, this is not assumed by the type.

A MapType is an enumeration with four possible values, directly corresponding to the four map types provided by Google.

The SQLiteDBMap class is an implementation of a MapProvider.
The class contains few fields, storing most of its data in an external SQLite database file.
When a call to GetImage is made on a SQLiteDBMap, it defers the work to three internal functions using the following psuedocode.
\begin{lstlisting}
fn GetImage(spec):
    cachedImage = GetCachedImage(spec)
    if cachedImage is not null:
        return cachedImage
    else:
        image = DownloadImage(spec)
        CacheImage(spec, image)
        return image
\end{lstlisting}
The above psuedocode ignores possible errors, such as network failures, that will be handled in the application.
The function DownloadImage makes a query on the Google Static Maps API,
while the functions CacheImage and GetCachedImage store and retrieve, respectively, images which have already been downloaded.

\subsubsection{Map Controls}
\quickfigure{0.85}{mapcontrol}{UML Diagram of Map Control Project and Dependencies}
The primary map control used to view a flight plan is the WorldMap class.
An instance of a WorldMap contains references to a FlightPlan and a MapProvider,
and has fields containing the coordinate of the display's center, as well as the current map type and zoom level.
The WorldMap class is strictly a view, and has no direct functionality for modifying data.

The WorldMap is implemented as a UserControl containing three subcontrols:
a canvas for drawing the map and the waypoints of the flight plan,
a control for selecting which map type is desired,
and a control for selecting the zoom level.
The FlightPlan is set as a DependencyProperty so that WPF is able to bind it to multiple locations, as it will be shared with the main application.

\subsubsection{Application}
\quickfigure{1.0}{application}{UML Diagram of Application Project and Dependencies}
MainPage is the entry point of the application, a Window which provides a tiled view (see \ref{sec:ui}) of three main components:
a FlightPlanChooser, a FlightPlanTable, and a WorldMap.
MainPage is also responsible for containing the peripheral user interface, such as a menu bar.
When a MainPage is constructed, it also constructs each component inside it recursively, during the process setting bindings as appropriate.
The most important binding to be aware of is the active FlightPlan, which must be synchronized correctly between the FlightPlanTable and the WorldMap.
The active FlightPlan is initially empty but not null,
and can be either modified in place through interaction with the FlightPlanTable or destroyed and replaced through the menu bar or the FlightPlanChooser.

The FlightPlanTable is a UserControl containing subcontrols for
  a textual, grid-based display of a flight plan as well
  as controls necessary to modify the active flight plan.
Modifications to the flight plan include adding or removing waypoints, reordering waypoints, or moving an existing waypoint.
Due to the DependencyProperty libraries, changes will automatically propogate across other components.

The FlightPlanChooser is a UserControl containing a view of all flight plans contained in the same folder as the open FlightPlan.
This control can be used to handle traditional file operations like opening, saving, and creating new files.
The menu bar and the FlightPlanChooser differ in functionality by intentional limitations on the FlightPlanChooser.
The FlightPlanChooser will only perform operations that follow the folder structure of a Garmin G1000-compatible SD card,
  while the menu bar will allow the user to save or load files from any path.

\subsection{Design Rationale}
The project structure chosen was maximally strict in an attempt to disallow incorrect usage and implementation.
A one-project design was briefly considered, but deemed less preferable to strong modulization due to increased ease of leaking implementation details.
Some classes and projects could easily be rearranged,
such as the FlightPlanChooser class which would also work in the MapControls project.
The FlightPlanChooser remains in the Application project instead because it contains sufficient application logic to not be a pure view component.

\newpage
\section{Human Interface Design} \label{sec:ui}
\subsection{Overview of User Interface}
The interface for the Flight Planning Editor will be a Windows desktop application with a design concept shown in the illustration below.
\quickfigure{0.95}{figs/FlightPlanning_Interface}{Mockup of User Interface}

\subsection{Screen Objects and Actions}
In the above window there are effectively four subsections to the interface. In the upper left corner, where it reads 'Lat', this portion of the interface is where the user can input both the lattitude and longitude coordinates as input into the flight editor. To add these coordinates to the current flight plan, simply click on the 'Add Waypoint' box and the program will create a new point on the map as well as create an addition into the Order of Points Visited section detailed below. \\
Directly below, there will be another section which reads the various flight plans that are currently loaded and their respective names (i.e. Flight Plan A). To distinguish which flight plan the user wants to current edit, that flight plan will have a box which shows an 'X' or is checked to denote that it is currently selected. \\
Underneath the above section, in the lower left corner of the application, there is a section for the order of points visited. In this section there will be a list of all the currently added waypoints named 'W1', 'W2'... Each of these waypoints are listed in order from top to the bottom. To the right of these names there are arrows pointing upwards and downwards, and by clicking on this icon up or down, the user will be able to re-arrange the order in which each point is visited and the application will swap the waypoint boxes in this section, as well as redraw the arrow on the map to reflect the changes made. In the upper left portion of each boxes for the waypoints will be a small box with an 'X' when clicked will delete the waypoint and remove it from the map.\\
The largest portion of the interface will display the map. This section will display the current area that the user has selected by panning and zooming in or out of the map. Inside there will be each of the waypoints listed and arrows denoting the direction of the flight plan. As the user enters a new lattitude and longitude, these waypoints will be refreshed and shown ontop of the map in their respective locations. \\
Also included, however not displayed in the diagram are the file and help pull down menus. Within the file menu, there will be the option to start a new or open an existing flight plan. The ability to save the flight plan, or to export onto the SD card. As well as being able to exit the application. Under the help menu, there will be a tutorial to showcase the basic functionality and an included about page with the company information and the version number of the application.

\newpage
\appendix

\bibliography{citations}{}
\bibliographystyle{plain}

\end{document}
